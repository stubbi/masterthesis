The thesis should study how well the RBM Ansatz from \cite{jnsson2018neuralnetwork} performs on 
the random circuit sampling experiments. It should give the reader an introduction to 
quantum computing and Restricted Boltzmann machines including their applications to quantum circuits.

The Random Circuit Sampling task will be justified as suited for quantum supremacy experiments and 
Google's results should be presented. Own experiments should be conducted for circuit sizes which
can still be verified using the Noctua cluster from the University of Paderborn \cite{noctua2020}.

If feasible, larger supercomputing facilities should be 
contacted to run experiments on more qubits than possible on the Nocuta cluster.

Additionaly, the netket library \cite{netket:2019} should be adapted to include RBM simulations of quantum circuits and be 
made available to the public as an open source project as part of the thesis.