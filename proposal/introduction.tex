Quantum computers have the potential to solve certain computational task exponentially faster
than classical computers \cite{feynman1999simulating,shor1999polynomial}. Demonstrating \textit{quantum supremacy} 
\cite{preskill2012quantum}, that is the moment a physical quantum computer outperforms classical 
computers on a given task for the first time, is a major step into the direction of realizing 
large scale quantum computers and currently a very active field of research \cite{google2019supremacy, Neill_2018, aaronson2016complexitytheoretic, boul2018quantum, aaronson2010computational, boixo2018characterizing}.

Last year, a team of researchers from Google claimed to have reached quantum supremacy with a 53 qubit
qauntum processor \cite{google2019supremacy}. However, as also quantum processors are not perfect but suffer from decoherent noise, the 
question persists if classical approximations of quantum computing can still challenge those results \cite{zhou2020limits}. 

Recently, neural networks have been successfully applied to the classical approximate simulation of quantum systems and
quantum computing \cite{Carleo_2017,jnsson2018neuralnetwork}. Applying them to quantum supremacy experiments therefore might give new insights about the
limits of classical simulations of quantum computing and the state of quantum supremacy.

