The following aspects of the thesis have been identified and define the working packages of the thesis.

\subsection{Research and Theory}
As a preparation of the thesis and a feasibility study, state of the art work has already been studied 
and understood on a good level of detail. Over the first 1.5 months (considering restricions mentioned below), further details will be clarified, including 
the proof of worst to average case reduction of random circuits, tensor network states and the fidelity results 
of the later and Google's quantum processor. New research in the field should be studied and included if applicable and feasible when it becomes published.

\subsection{Implementation}
In a prestudy, the adaption of the netket library has already been started and prepared to run circuits with 
the gate set of $X$, $Y$, $Z$, $H$, $RZ(\theta)$ and $Z(\theta)$ on circuits defined in the QASM language \cite{cross2017open}. For the RCS experiments, $sqrtX$ and $sqrtY$ which work similar to the Hadamard gate are still 
needed. The implementation of these gates including a testing phase is expected to take one more week and should
be done next.

\subsection{Experiments}
First experimets for random ciruit sampling have already been setup and run on noctua to verify the feasibility of
the setup. When the implementation of the RBM Ansatz is finished, new experiments should be started, adapet and be run in the 
background while further undertanding the theoretic backgrounds mentioned above. Running the experiments is expected 
to take 1.5 months with little manual work. The maintenance of noctua in mid of March has to be considered
and will extend the pure time allocated for the experiments by two weeks.

Another two weeks are reserved for the evaluation and discussion of the results with my supervisors.

\subsection{Writing}
The introductary chapters of the thesis will be written either when time is left during the phase of further 
understanding of the theoretic background or afterwards. When the experiments finished, the results should be written
down. Thus the main writing phase of the thesis should start in 2 months and be finished after another 2 months.

The last weeks shold be reserved for proof reading and fine tuning of the thesis. 

\subsection{Defense}
The defense of the thesis should either happen in the planned seminar of thesis students in the quantum computer
science departement either at the end or as soon as the results are available. The latter option will allow to include 
results of potential discussions into the thesis' evaluation sections.

Another alternative to the students seminar will be the inderdisciplinary quantum networks seminar if free slots are available. The preperation 
of the defense is considered to take up to one week of work. This buffer is included in the research and proof reading
phases.

\subsection{Summary}
So in summary, the timeline for the thesis will bee:

\begin{enumerate}
    \item Finish Implementation (1 week)
    \item Experiments setup (1 week)
    \item \textit{Running Experiments (8 weeks in the background)}
    \item Further research and Theory (6 weeks)
    \item Evaluation (3 weeks)
    \item Writing (6 weeks)
    \item Defense (1 week)
    \item Proof reading and finishing up (2 weeks)
\end{enumerate}