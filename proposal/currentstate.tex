Exact classical simulations of quantum systems require up to exponential computational resources \cite{feynman1999simulating}.
Harnessing the characteristics of quantum physics for computation therefore opens the possibility to build quantum computers which can outperform classical computers on 
certain computational tasks \cite{shor1999polynomial}. The moment when a physical quantum computer outperforms any classical computer on a well defined task
for the first time has been coined \textit{quantum supremacy} by John
Preskill in 2012 and is currently a very active field of research.

Different proposals for quantum supremacy experiments like Boson Sampling \cite{aaronson2010computational}, Fourier Transform \cite{fefferman2015power} and Random Circuit Sampling (RCS) \cite{boixo2018characterizing} have
been given and theoretically analyzed. With recent progress in building quantum computers based on 
superconducting qubits, Random Circuit Sampling became the most promising to be physically realized in the near future \cite{boul2018quantum}.

The proposal for RCS is based on the fact that the output distributions of random circuits of specific structures will 
approximate the Porter-Thomson distribution and are prone to single gate errors. By that they are not only hard 
to simulate classically but the cross entropy fidelity of the classical simulation and the output of a 
quantum processor also acts as a benchmark for the fidelity of the quantum device \cite{boixo2018characterizing}. Complexity theoretic 
evidence for a worst to average case reduction completes the theory of 
RCS as a well defined quantum supremacy experiment \cite{boul2018quantum}.

Recently, a research team from Google performed RCS on a 53 qubits superconducting 
quantum processor and claimed to have demonstrated quantum supremacy \cite{google2019supremacy}. Nevertheless, a team from IBM 
claimed that the classical exact simulations of the random circuits could run within four days
rather than the reported 10,000 years by the Google team thus questioning the claim \cite{pednault2019leveraging}.

As quantum processors will not be perfect but suffer from decoherent noise, perfect simulations 
of the quantum circuits are not even necessary to challenge the quantum supremacy claims. 
As long as physical quantum computers can not outperform classical approximate simulations in 
terms of noise, quantum supremacy has not been reached \cite{boixo2018characterizing}.
Indeed, Tensor Network states produce a similar fidelity to the one from Google's quantum processor on the RCS task 
with overhead only polynomial in the circuits size and depths, thus questioning the claim for quantum supremacy even further \cite{fefferman2015power}.

While Tensor Network states are a known classical approximation technique for quantum systems for 
some time, recently neural networks gained popularity as a new Ansatz for the classical simulation
of quantum physics. Carleo applied Restricted Boltzmann Machines (RBM) to learn the wavefunctions of many body 
quantum systems \cite{Carleo_2017}. Afterwards, Gao could give theoretic proof that while General 
Boltzmann machines are able to represent quantum states exactly but make the sampling process P\#
hard, RBMs can approximate any quantum system with a worst case exponential number of hidden units while keeping the 
sampling runtime efficient \cite{Gao_2017}.

In 2018, Jónsso applied RBMs to the classical simulation of 
quantum circuits with a gate fidelity of $10^{-3}$ \cite{jnsson2018neuralnetwork}. This opens the question of which 
cross entropy fidelities on the RCS experiments can be reached by RBMs and thus how they compare to 
Tensor Network states and Google's quantum processor.