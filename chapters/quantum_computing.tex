\chapter{Quantum Computing}

This chapter gives an introduction to the field of \textit{Quantum Computation}. It covers the 
basics of qubits, quantum gates and circuits and a high level overview of quantum complexity classes. This chapter 
is based on \cite{nielsen2002quantum} which is recommended as a reference for a deeper introduction 
into the field. 

\section{Qubits}

While classical computers harness classical physical phenomena like 
electrical current to perform calculations, quantum computers harness the properties of 
\textit{quantum mechanics}. The simplest quantum mechanical system is a quantum bit, or
\texit{qubit} for short. The qubit is the fundamental building block of a quantum computer. Its
properties will be explained in this section.

Mathematically, a qubit can be described by a two dimensional complex valued unit vector,
also called the \textit{state vector} or \textit{wave function} of the qubit:

\begin{equation}
  \label{eq:qubit1}
  \Ket{\psi} = \begin{pmatrix} \alpha \\ \beta \end{pmatrix}
\end{equation}

with \texit{amplitudes} $\alpha, \beta \in \mathbb{C}$ and $\left\alpha\right^2 + \left\beta\right^2 =1$. 
Rewriting $\Ket{\psi}$ as

\begin{equation}
  \Ket{\psi} = e^{i \gamma} (\cos{\frac{\theta}{2}} \Ket{0} + e^{i \rho} \sin{\frac{\theta}{2}} \Ket{1})
\end{equation}

with $\theta, \rho, \gmma \in \mathbb{R}$
allows for an intuitive interpration of a qubit as a point on the surface of the three dimensional unit sphere,
also called the \textit{Bloch Sphere}, which is visualised in figure
\ref{fig:blochsphere}. The factor $e^{i \gamma}$ in front is called a
\textit{global phase} and can be ignored:

\begin{equation}
  \Ket{\psi} = \cos{\frac{\theta}{2}} \Ket{0} + e^{i \rho} \sin{\frac{\theta}{2}} \Ket{1}
\end{equation}

\begin{figure}[H]
  \label{fig:blochsphere}
  \centering
  \begin{tikzpicture}

    % Define radius
    \def\r{3}

    % Bloch vector
    \draw (0,0) node[circle,fill,inner sep=1] (orig) {} -- (\r/3,0.7*\r) node[circle,fill,inner sep=0.7,label=right:$\Ket{\psi}$] (a) {};
    \draw[dashed] (orig) -- (\r/3,-\r/5) node (phi) {} -- (a);

    % Sphere
    \draw[line width=0.5mm] (orig) circle (\r);
    \draw[dashed, line width=0.5mm] (orig) ellipse (\r{} and \r/5);

    % Axes
    \draw[->] (orig) -- ++(-\r/2,-\r/3) node[below] (x) {$x$};
    \draw[->] (orig) -- ++(\r+0.5,0) node[right] (y) {$y$};
    \draw[->] (orig) -- ++(0,\r+0.5) node[above] (z) {$z$};

    % Basis States
    \filldraw[fill=white, line width=0.3mm] (0,\r) circle (0.15) node[label={[label distance=0.01mm]135:$\Ket{0}$}] (0);
    \filldraw[fill=white, line width=0.3mm] (0,-\r) circle (0.15) node[label={[label distance=0.01mm]325:$\Ket{1}$}] (1);

    % Angles
    \pic [draw=gray,text=gray,--,"$\phi$", angle radius=0.5cm] {angle = x--orig--phi};
    \pic [draw=gray,text=gray,--,"$\theta$", angle radius=1cm] {angle = a--orig--x3};

  \end{tikzpicture}
  \caption{The Bloch Sphere in its full beauty.}
\end{figure}

The states on the north and south pole of the Bloch Sphere are called the
\textit{computational basis states} defined as:

\begin{equation}
  \Ket{0} = \begin{pmatrix} 1 \\ 0 \end{pmatrix}
\end{equation}

and

\begin{equation}
  \Ket{1} = \begin{pmatrix} 0 \\ 1 \end{pmatrix}.
\end{equation}

These states correspond to the classical states $0$ and $1$ of a classical bit. Thus, yet another way of writing the wave
function of a qubit, which is also the most practical one for quantum computation, is as a linear combination of the
two computational basis states:

\begin{equation}
  \Ket{\psi} = \alpha \cdot \Ket{0} + \beta \cdot \Ket{1}.
\end{equation}

A qubit whose amplitudes $\alpha$ and $\beta$ are both non-zero is in a so called \textit{superposition} of the two
computational basis state. A qubit in a superposition can be interpreted as
 being in both states $\Ket{0}$ and $\Ket{1}$ at the same time.
This property is one of the underlying reason why the classical simulation of
quantum systems is computationally so hard and 
one of the sources of the potential computational power of quantum computers.
It is not possible to read the values of the amplitudes $\alpha$ and $\beta$
directly, though.

The qubit can only be in a superposition as long as it is completely isolated
from its environment. Such a state is called a \textit{coherent} state. As
soon as the qubit interacts with its environment, as it is necessary in order to read
its value, it collapeses randomly into one of the two computational basis
states. All the information stored in the amplitudes before gets lost in this process.

In a quantum computer the qubits are in a coherent state and potentially in a superposition during the computation. At the end of the computation, a so called
\textit{projective measurement} is performed to read the states of the individual qubits. A projective measurement is given
by \textit{operators}, in our case $M_0 = \Ket{0}\Bra{0}$ and $M_1 = \Ket{1}\Bra{1}$. The probablities to observe a qubit in the $\Ket{0}$ or
$\Ket{1}$ state are then defined by the amplitudes

\begin{align}
  p(0) &= \Bra{\psi} M_0^{\dagger}M_0 \Ket{\psi} \\
       &=  \Bra{\psi} M_0 \Ket{\psi} \\
       &= \alpha \Bra{0} \Ket{0} \Bra{0} \alpha \Ket{0} + \beta \Bra{1} \Ket{0} \Bra{0} \beta \Ket{1} \\
       &= \alpha^2 \Bra{0} \Ket{0} \Bra{0} \Ket{0} + \beta^2 \Bra{1} \Ket{0} \Bra{0} \Ket{1} \\
       &= \alpha^2
\end{align}

and

\begin{align}
  p(1) &= \Bra{\psi} M_1^{\dagger} M_1 \Ket{\psi} \\
       &= \Bra{\psi} M_1 \Ket{\psi} \\
       &= \alpha \Bra{0} \Ket{1} \Bra{1} \alpha \Ket{0} + \beta \Bra{1} \Ket{1} \Bra{1} \beta \Ket{1} \\
       &= \alpha^2 \Bra{0} \Ket{1} \Bra{1} \Ket{0} + \beta^2 \Bra{1} \Ket{1} \Bra{1} \Ket{1} \\
       &= \beta^2
\end{align}

as $M^{\dagger}M = M$ and $\Bra{i} \Ket{j} = \delta_{ij}$.

After the measurement, the state of the qubit will be either

\begin{equation}
  \Ket{\psi\dagger} = \frac{M_0 \Ket{\psi}}{\sqrt{p(0)}}
\end{equation}

or

\begin{equation}
  \Ket{\psi\dagger} = \frac{M_1 \Ket{\psi}}{\sqrt{p(1)}}
\end{equation}

respectively, implying that the state of the qubit collapsed into one of the two computational
basis states. Thus, any succinct measurement will result in the same outcome.
In order to perform multiple measurements on the same state, it has to be prepared multiple times.

Two special qubit states which often occur in the domain of quantum computation are the $\Ket{-}$ and $\Ket{+}$ state defined as

\begin{equation}
  \Ket{-} = \frac{1}{\sqrt{2}} \cdot \Ket{0} - \frac{1}{\sqrt{2}} \cdot \Ket{1}
\end{equation}

and

\begin{equation}
  \Ket{+} = \frac{1}{\sqrt{2}} \cdot \Ket{0} + \frac{1}{\sqrt{2}} \cdot \Ket{1}.
\end{equation}

Those states differ in a \textit{relative phase} and both have the same measurement statistics, lying between the states $\Ket{0}$ and
$\Ket{1}$ on the Bloch Sphere:

\begin{align}
  p(0) &= \Bra{+} M_0 \Ket{+} \\
       &= \frac{1}{\sqrt{2}} \Bra{0} \Ket{0} \Bra{0} \frac{1}{\sqrt{2}} \Ket{0} + \frac{1}{\sqrt{2}} \Bra{1} \Ket{0} \Bra{0} \frac{1}{\sqrt{2}} \Ket{1} \\
       &= \frac{1}{2} \Bra{0} \Ket{0} \Bra{0} \Ket{0} + \frac{1}{2} \Bra{1} \Ket{0} \Bra{0} \Ket{1} \\
       &= \frac{1}{2}
\end{align}


and

\begin{align}
  p(1) &= \Bra{+} M_1 \Ket{+} \\
       &= \frac{1}{\sqrt{2}} \Bra{0} \Ket{1} \Bra{1} \frac{1}{\sqrt{2}} \Ket{0} + \frac{1}{\sqrt{2}} \Bra{1} \Ket{1} \Bra{1} \frac{1}{\sqrt{2}} \Ket{1} \\
       &= \frac{1}{2} \Bra{0} \Ket{1} \Bra{1} \Ket{0} + \frac{1}{2} \Bra{1} \Ket{1} \Bra{1} \Ket{1} \\
       &= \frac{1}{2}
\end{align}



for the $\Ket{+}$ state. The same probablities hold for the
$\Ket{-}$ state.
Again, the state will be destroyed on measurement and be either $\Ket{0}$ or $\Ket{1}$ afterwards, depending on the measurement outcome.

A qubit represents the simplest quantum system. The stochastic nature of quantum physics allows a qubit to be in a superposition of the
two computational basist states $\Ket{0}$ and $\Ket{1}$. 
Nevertheless, it is not possible to access such a superposition state directly
but only ever to observe a qubit in one of the two computational basis states.

The next section will describe how combining multiple qubits into a bigger
quantum system gives raise to an exponential growth of the state vector of such systems and how multiple qubits can be \textit{entangeled}
with each other.

\section{Multiple Qubits and Entanglement}

The state vector of a single qubit lies in a two dimensional 
space, assigning a complex valued amplitude to each of the both possible measurement outcomes
$\Ket{0}$ and $\Ket{1}$. The state vector of a multi qubit system is defined by
the \textit{tensor product} of the state vectors of the individual subsystems.
For a two qubit system consisting of $\Ket{\psi_1} = \alpha_1 \Ket{0} + \beta_1
\Ket{1}$ and $\Ket{\psi_2} = \alpha_2 \Ket{0} + \beta_2 \Ket{1}$ the state space
is given by:

\begin{align}
  \Ket{\psi_{1,2}} &= \Ket{\psi_1} \otimes \Ket{\psi_2} \\
                   &= (\alpha_1 \Ket{0} + \beta_1 \Ket{1}) \otimes (\alpha_2 \Ket{0} + \beta_2 \Ket{1}) \\
                   &= \alpha_1 \alpha_2 \Ket{0}\Ket{0} + \alpha_1 \beta_2 \Ket{0} \Ket{1} + \beta_1 \alpha_2 \Ket{1} \Ket{1} + \beta_1 \beta_2 \Ket{1} \Ket{1} \\
                   &= \alpha_1 \alpha_2 \Ket{00} + \alpha_1 \beta_2 \Ket{01} + \beta_1 \alpha_2 \Ket{10} + \beta_1 \beta_2 \Ket{11} \\
                   &= \alpha_1 \alpha_2 \Ket{0} + \alpha_1 \beta_2 \Ket{1} + \beta_1 \alpha_2 \Ket{2} + \beta_1 \beta_2 \Ket{3}
\end{align}

where we define $\Ket{0} = \Ket{00}$, $\Ket{1} = \Ket{01}$, $\Ket{2} = \Ket{10}$
and $\Ket{3} = \Ket{11}$. In the general case, the wave function of a
n-dimensional qudit system is therefore given by a $2^n$ dimensional vector:

\begin{equation}
  \Ket{\psi} = \sum_{i=0}^{2^d}(a_i \Ket{i}).
\end{equation}

A classical simulation of a d-dimensional qudit system therefore has to keep
track of $2^d$ complex amplitudes. This makes it impossible to simulate quantum
systems above a certain size. For the classical simulation of a system
consisting of 500 qubits the dimension of the state vector is already larger
than the number of atoms in the universe.  

As in the single qubit case, the probability to observe state $\Ket{n}$ is
given by $a_n^2$. It is also possible to only measure a subset of the qubits,
leaving the other qubits in the normalised state:

\begin{equation}
  \Ket{\psi\prime} = \frac{(M_m \otimes I) \Ket{\psi}}{\sqrt(p(m))}.
\end{equation}

An interesting property that can appear in multi qudit systems is
\textit{entanglement}. Consider for instance the two qudit state

\begin{equation}
  \Ket{\psi} = \frac{\Ket{00} + \Ket{11}}{\sqrt{2}}.
\end{equation}

which can be created with a very simple quantum program as shown later in
section \ref{sec:...}.
Note that this state is a proper two qubit state as it fulfills the
normalisation property
$a_{00}^2 + a_{11}^2 = \frac{1}{\sqrt{2}}^2 + \frac{1}{\sqrt{2}}^2 = \frac{1}{2}
+ \frac{1}{2} = 1$.

Nevertheless, there are no two single qubit states $\Ket{a}$
and $\Ket{b}$ such that $\Ket{\psi} = \Ket{a} \Ket{b}$. The two qubits of
$\Ket{\psi}$ are \textit{entangled} with each other. Measuring one of the qubits
immediately determines the state of the other qubit, i.e. measuring the first
qubit as $\Ket{0}$ happens with probability:

\begin{align}
  p_1(0) &= \Bra{\psi} (M_0 \otimes I)^{\dagger} (M_o \otimes I) \Ket{\psi} \\
         &= \frac{1}{\sqrt{2}} \Bra{00} \frac{1}{\sqrt{2}} \Ket{00} \\
         &= \frac{1}{2}
\end{align}

leaving the system in the post measurement state

\begin{align}
  \Ket{\psi\prime} &= \frac{(M_0 \otimes I) \psi}{\sqrt{p_1(0)}} \\
                   &= \frac{\frac{1}{\sqrt{2}} \Ket{00}}{\frac{1}{\sqrt{2}}} \\
                   &= \Ket{00}.
\end{align}

The same maths apply for the case of measuring the first qubit as $\Ket{1}$.

Even if the state of the second qubit is not determined before, it will be
either $\Ket{0}$ or $\Ket{1}$ from the moment on the \textit{other} qubit has been measured.
The exponential growth of the state space dimesion and entanglement are two
important properties of quantum systems which make them hard to simulate
classicaly.
Known quantum algorithms with quantum speedups like Shor's algorithm
create entanglement in clever ways to perform calculations in ways which seem
to be impossible for classical computers.

The next section introduces the notion of \textit{quantum gates} which allow to
manipulate the state of a single or multiple qubits.

\section{Quantum Gates}

It is physically possible to modify the state vector of a quantum system even
when it is in a coherent state. This opens the theoretical possibility to perform
calculations with them and build quantum computers.

Mofifications happen by the application of so called \textit{quantum gates} to
the state. Quantum physics allows the gates only to be linear and reversable. Thus,
quantum gates can be mathematically described as unitary matrices. Gates vary in the number of
qubits on which they act as well as in the effects they have on the qubits' state.

This chapter will introduce single and multi qubit gates and gives an overview
over some common quantum operations.

\subsection{Single Qubit Gates}

Single qubit gates are linear mappings that can be applied to the state vector
of a single qubit. Mathematically, single qubit gates can be described by $2
\times 2$ unitary matrices. The fact that these matrices are unitary makes sure
the state $\Ket{\phi} = G \Ket{\psi}$ after gate $G$ has been applied to state $\Ket{\psi}$ is a
proper quantum state which preserves the normalisation constraint
$\alpha_{\Ket{\phi}}^2 + \beta_{\Ket{\phi}}^2 = 1$.

An example of a single qubit gate is the NOT or $X$ gate, also denoted as:

\begin{figure}[h]
  \centering
  \leavemode
  \mbox{
    \Qcircuit @C=1em @R=1em {
      & \gate{X} & \qw
    }
  }
\end{figure}

or

\begin{figure}[h]
  \centering
  \leavemode
  \mbox{
    \Qcircuit @C=1em @R=1em {
      & \targ & \qw
    }
  }
\end{figure}

The $X$ gate is the quantum analoge of
the classical NOT gate. Similar to the classical version, the $X$ gate swaps the
states $\Ket{0}$ and $\Ket{1}$ when applied to them. It is even more general though, as it maps a
single qubit state of the form

\begin{equation}
    \Ket{\psi} = \alpha \Ket{0} + \beta \Ket{1}
\end{equation}

to the state 

\begin{equation}
  \Ket{\phi} = \beta \Ket{0} + \alpha \Ket{1}
\end{equation}

thus swapping the amplitudes for $\Ket{0}$ and $\Ket{1}$. The $X$ gate has the
following matrix representation:

\begin{equation}
  X =
  \begin{pmatrix}
    0 & 1 \\
    1 & 0
    \end{pmatrix}
\end{equation}

A single qubit gate is applied to a quantum state by matrix multiplication:

\begin{align}
  \Ket{\phi} &= X \Ket{\psi} \\
             &=
               \begin{pmatrix}
                 0 & 1 \\
                 1 & 0
               \end{pmatrix}
                \Ket{\psi} \\
             &= \begin{pmatrix}
                 0 & 1 \\
                 1 & 0
               \end{pmatrix}
                \begin{pmatrix}
                  \alpha \\
                  \beta
                \end{pmatrix} \\
             &= \begin{pmatrix}
                 \beta \\
                 \alpha
               \end{pmatrix}
\end{align}

Another example for a single qubit gate with no classical analoge is the
\textit{Hadamard gate} or $H$ gate, described by the unitary:

\begin{equation}
  H = \frac{1}{\sqrt{2}}
  \begin{pmatrix}
    1 & 1 \\
    1 & -1
  \end{pmatrix}
\end{equation}

Notice that the H gate is a unitary as it is reversable:

\begin{equation}
   H^{\dagger} H = I 
\end{equation}

The Hadamard gate maps between the so called $Z$ and $X$ bases, mapping the
states:

\begin{align}
  H \Ket{0} &= \Ket{+} \\
  H \Ket{1} &= \Ket{-}
\end{align}

and back

\begin{align}
  H \Ket{+} &= \Ket{0} \\
  H \Ket{-} &= \Ket{1}.
\end{align}

The Hadamard gate is also a good example of how the Bloch Sphere visualisation can help to understand the 
transformation of a qubit state. The application of the Hadamard gate to the $\Ket{+}$ is shown in figure X.


\begin{figure}[H]
  \centering
  \begin{subfigure}[b]{0.3\textwidth}
    \centering
\begin{tikzpicture}

    % Define radius
    \def\r{2}

    % Bloch vector
    \draw (0,0) node[circle,fill,inner sep=1] (orig) {} -- (\r/3,0.7*\r) node[circle,fill,inner sep=0.7,label=right:$\Ket{\psi}$] (a) {};
    \draw[dashed] (orig) -- (\r/3,-\r/5) node (phi) {} -- (a);

    % Sphere
    \draw[line width=0.5mm] (orig) circle (\r);
    \draw[dashed, line width=0.5mm] (orig) ellipse (\r{} and \r/5);

    % Axes
    \draw[->] (orig) -- ++(-\r/2,-\r/3) node[below] (x) {$x$};
    \draw[->] (orig) -- ++(\r+0.5,0) node[right] (y) {$y$};
    \draw[->] (orig) -- ++(0,\r+0.5) node[above] (z) {$z$};

    % Basis States
    \filldraw[fill=white, line width=0.3mm] (0,\r) circle (0.15) node[label={[label distance=0.01mm]135:$\Ket{0}$}] (0);
    \filldraw[fill=white, line width=0.3mm] (0,-\r) circle (0.15) node[label={[label distance=0.01mm]325:$\Ket{1}$}] (1);

    % Angles
    \pic [draw=gray,text=gray,--,"$\phi$", angle radius=0.5cm] {angle = x--orig--phi};
    \pic [draw=gray,text=gray,--,"$\theta$", angle radius=1cm] {angle = a--orig--x3};

  \end{tikzpicture}

    \caption{$y=x$}
    \label{fig:y equals x}
  \end{subfigure}
  \hfill
  \begin{subfigure}[b]{0.3\textwidth}
    \centering
\begin{tikzpicture}

    % Define radius
    \def\r{2}

    % Bloch vector
    \draw (0,0) node[circle,fill,inner sep=1] (orig) {} -- (\r/3,0.7*\r) node[circle,fill,inner sep=0.7,label=right:$\Ket{\psi}$] (a) {};
    \draw[dashed] (orig) -- (\r/3,-\r/5) node (phi) {} -- (a);

    % Sphere
    \draw[line width=0.5mm] (orig) circle (\r);
    \draw[dashed, line width=0.5mm] (orig) ellipse (\r{} and \r/5);

    % Axes
    \draw[->] (orig) -- ++(-\r/2,-\r/3) node[below] (x) {$x$};
    \draw[->] (orig) -- ++(\r+0.5,0) node[right] (y) {$y$};
    \draw[->] (orig) -- ++(0,\r+0.5) node[above] (z) {$z$};

    % Basis States
    \filldraw[fill=white, line width=0.3mm] (0,\r) circle (0.15) node[label={[label distance=0.01mm]135:$\Ket{0}$}] (0);
    \filldraw[fill=white, line width=0.3mm] (0,-\r) circle (0.15) node[label={[label distance=0.01mm]325:$\Ket{1}$}] (1);

    % Angles
    \pic [draw=gray,text=gray,--,"$\phi$", angle radius=0.5cm] {angle = x--orig--phi};
    \pic [draw=gray,text=gray,--,"$\theta$", angle radius=1cm] {angle = a--orig--x3};

  \end{tikzpicture}

    \caption{$y=3sinx$}
    \label{fig:three sin x}
  \end{subfigure}
  \hfill
  \begin{subfigure}[b]{0.3\textwidth}
    \centering
\begin{tikzpicture}

    % Define radius
    \def\r{2}

    % Bloch vector
    \draw (0,0) node[circle,fill,inner sep=1] (orig) {} -- (\r/3,0.7*\r) node[circle,fill,inner sep=0.7,label=right:$\Ket{\psi}$] (a) {};
    \draw[dashed] (orig) -- (\r/3,-\r/5) node (phi) {} -- (a);

    % Sphere
    \draw[line width=0.5mm] (orig) circle (\r);
    \draw[dashed, line width=0.5mm] (orig) ellipse (\r{} and \r/5);

    % Axes
    \draw[->] (orig) -- ++(-\r/2,-\r/3) node[below] (x) {$x$};
    \draw[->] (orig) -- ++(\r+0.5,0) node[right] (y) {$y$};
    \draw[->] (orig) -- ++(0,\r+0.5) node[above] (z) {$z$};

    % Basis States
    \filldraw[fill=white, line width=0.3mm] (0,\r) circle (0.15) node[label={[label distance=0.01mm]135:$\Ket{0}$}] (0);
    \filldraw[fill=white, line width=0.3mm] (0,-\r) circle (0.15) node[label={[label distance=0.01mm]325:$\Ket{1}$}] (1);

    % Angles
    \pic [draw=gray,text=gray,--,"$\phi$", angle radius=0.5cm] {angle = x--orig--phi};
    \pic [draw=gray,text=gray,--,"$\theta$", angle radius=1cm] {angle = a--orig--x3};

  \end{tikzpicture}

    \caption{$y=5/x$}
    \label{fig:five over x}
  \end{subfigure}
  \caption{Application of the Hadamard gate visualised in the Bloch Sphere.}
  \label{fig:three graphs}
\end{figure}

in figure \ref{fig:singlegates}.

\begin{table}[H]
  \centering
  \begin{tabular}{c c c}
    Operator & Gate & Matrix \\[20pt]
    $X$ &  \Qcircuit @C=1em @R=.7em { & \gate{X} & \qw } & \begin{pmatrix} 0 & 1 \\ 1 & 0\end{pmatrix} \\[20pt]
    $Y$ &  \Qcircuit @C=1em @R=.7em { & \gate{Y} & \qw } & \begin{pmatrix} 0 & -i \\ i & 0\end{pmatrix} \\[20pt]
    $Z$ &  \Qcircuit @C=1em @R=.7em { & \gate{Z} & \qw } & \begin{pmatrix} 1 & 0 \\ 0 & -1\end{pmatrix} \\[20pt]
    $H$ &  \Qcircuit @C=1em @R=.7em { & \gate{H} & \qw } & $\frac{1}{\sqrt{2}}$ \begin{pmatrix} 1 & 1 \\ 1 & -1\end{pmatrix} \\[20pt]
    $S$ &  \Qcircuit @C=1em @R=.7em { & \gate{S} & \qw } & \begin{pmatrix} 1 & 0 \\ 0 & i\end{pmatrix} \\[20pt]
    $T$ &  \Qcircuit @C=1em @R=.7em { & \gate{T} & \qw } & \begin{pmatrix} 1 & 0 \\ 0 & e^{i\frac{\pi}{2}}\end{pmatrix} \\[20pt]
  \end{tabular}
\end{table}

Though there are indefinitely many possible quantum gates in theory, 
a universal finite gate set 
consisting of only a few gates is sufficient to construct arbitrary unitary
transformations. This is the same as in the classical case where NAND gates
suffice to build up arbitrary classical computation circuits and 
 good news for the facrication of quantum computers with reusable components.

However, single qubit gates are not sufficient to build all possible unitary transformations on multi qubit systems. 
For this purpose, multi qubit gates are a necessity.

\subsection{Multi Qubit Gates}

Single qubit gates are not enough to create universal gate sets for quantum
computation. In order to run arbitrary quantum programs 
 \textit{controlled} gates are needed.
Controlled quantum gates are simple extensions of single qubit gates. Every single qubit gate can be implemented as a controlled 
version of it with one \textit{control} and one \textit{target} qubit. Only if
the control qubit is in the $\Ket{1}$ state, the gate is applied to the target
qubit. As the control qubit can be in a superposition state, it is possible 
to apply and not apply the gate to the target qubit at the same time, so
to say.

The propotypical two qubit gate is the controlled NOT or CNOT gate. It is the controlled version of the 
$X$ gate described before. As the CNOT gate acts on a two qubit state, it can be described by 
a $4 \times 4$ matrix. Each column describes the mapping of one of the four base
state $\Ket{00}$, $\Ket{01}$, $\Ket{10}$ and $\Ket{11}$. 
The matrix representation of the CNOT gate is the following:

\begin{equation}
  CNOT = \begin{pmatrix}
    1 & 0 & 0 & 0 \\
    0 & 1 & 0 & 0 \\
    0 & 0 & 0 & 1 \\
    0 & 0 & 1 & 0
    \end{pmatrix}
\end{equation}

The matrix can be read as follows: The first two colums describe that the vectors $\Ket{00}$ and $\Ket{01}$ won't change 
when the gate is applied to these states. Only when the first qubit is in the $\Ket{1}$ state, the state of the second qubit 
will be swapped, thus $\Ket{10}$ will be mapped to $\Ket{11}$ and $\Ket{11}$ to $\Ket{10}$. 

This also shows how the matrix representation of generalised two qubit gates can be derived: As long as the 
control qubit is off, the gate has no effect on the state. Thus, the matrix for
the controlled gate is the same as the identity with 1s on the 
diagonal and 0s elsewhere in the upper left corner. Only when the first qubit is in the $\Ket{1}$ state, the matrix differs from the identity. For example, the 
controlled version of the Z gate

\begin{equation}
  Z = \begin{pmatrix}
    1 & 0 \\
    0 & -1
    \end{pmatrix}
\end{equation}

is given by:

\begin{equation}
    CZ = \begin{pmatrix}
      1 & 0 & 0 & 0 \\
      0 & 1 & 0 & 0 \\
      0 & 0 & 1 & 0 \\
      0 & 0 & 0 & -1
      \end{pmatrix}.
\end{equation}

Controlled gates are represented with a vertical line ending in a black dot
indicating the control qubit:

\begin{figure}[h]
  \centering
  \leavemode
  \mbox{
    \Qcircuit @C=1em @R=1em {
      &  \ctrl{1} & \qw \\
      &  \targ{} & \qw
    }
  }
\end{figure}

With two qubit operations at hand, it is possible to build a universal gate set. A commonly used gate set 
is the so called \textit{Clifford + T} set, consists of the gates $CNOT$, $H$, $S$ and
$T$. The Solovay-Kitaev theorem guarantees that this set can efficiently
approximate any unitary operation. In this study, another universal gate set will be used though, 
composed of the $CZ$, $\sqrt{X}$, $\sqrt{Y}$ and $T$ gates.

The next sections will demonstrate how gates can be composed into \textit{quantum
circuits}, a notation for quantum programs.

\section{Quantum Circuits}

The standard model for computation in theoretical computer science is the Turing Machine. Inventend by 
Alan Turing in 1936 during World War Two, it is a powerful tool to understand the limits of (classical) 
computation. The
theoretical nature of the Turing machine gives it unlimited computational
resources like memory though, which do not reflect the properties of realisable
computer architectures.

Another model of computation which does not suffer from this gap between theory
and practical devices is the \textit{circuit model}. In the classical circuit model, each bit is represented by a wire and operations (or gates) are represented by different shapes acting on those wires. The circuit model is as powerful as a Turing Machine and the 
model of choice in the field of quantum computing.

Quantum programs are described by quantum circuits. As in the classical case, each qubit is represented 
by a wire and each gate by a rectangular on those wires. Figure
\ref{fig:circuit0} describes a very simple quantum circuit which flips a single
qubit starting in the $\Ket{0}$ state by applying a $X$ gate to it before
measuring it.

\begin{figure}[h]
  \centering
  \leavemode
  \mbox{
    \Qcircuit @C=1em @R=1em {
      &  \lstick{\Ket{0}} & \qw & \gate{$X$} & \qw & \meter{0/1} 
    }
  }
  \label{fig:circuit0}
  \caption{A simple quantum circuit.}
\end{figure}

The circuit is being read from left to right. Important to note though is that
for the calcuation of the resulting state, the matrix representation of the
gates are multiplied from the left side to the current state by simply following
the rules of linear algebra.
In the example above the final state of the program represented at the end of the circuit is:

\begin{align}
  \Ket{\phi} &= X \Ket{\psi} \\
             &=
               \begin{pmatrix}
                 0 & 1 \\
                 1 & 0
               \end{pmatrix}
                     \Ket{\psi} \\
             &= \begin{pmatrix}
               0 & 1 \\
               1 & 0
             \end{pmatrix}
                   \begin{pmatrix}
                     \alpha \\
                     \beta
                   \end{pmatrix} \\
             &= \begin{pmatrix}
               \beta \\
               \alpha
             \end{pmatrix}.
\end{align}

In practice, circuits will consist of several qubits and multiple gates applied
to them. It is possible to apply gates to different qubits sequentially as well
as in parallel:

\begin{figure}[h]
  \centering
  \leavemode
  \mbox{
    \Qcircuit @C=1em @R=1em {
      & \lstick{\Ket{0}} & \gate{H} & \gate{Z} & \qw & \gate{H} & \qw & \meter{0/1} \\
      & \lstick{\Ket{0}} & \qw & \gate{X} & \qw & \ctrl{-1} & \qw & \meter{0/1}
    }
  }
  \label{fig:circuit1}
  \caption{A slightly more complex quantum circuit.}
\end{figure}

Note that when calculating the state of the quantum program above, one has to
build the tensor product of all qubit states to calculate the 
state of the system. For the circuit from figure \ref{fig:circuit1}, the initial state before any gate has been applied 
is:

\begin{align}
    \Ket{\psi_{init}} &= \Ket{0} \otimes \Ket{0} \\
                      &= \begin{pmatrix} 1 \\ 0 \end{pmatrix} \otimes \begin{pmatrix} 1 \\ 0 \end{pmatrix} \\
                      &= \begin{pmatrix} 1 \\ 0 \\ 0 \\ 0 \end{pmatrix} \\
                      &= \Ket{00} 
\end{align}

When a gate gets applied to only a subset of the qubits as in the case of the
first $H$ gate in the circuit above, the unitary applied to the multi qubit state is
implicitely the tensor product of the gate on that qubit and identity matrices on the
remaining qubits. For the given circuit, the state $\Ket{\psi_{H_1}}$ after the
first $H$ gate on the first qubit is calculated as: 

\begin{align}
  \Ket{\psi_{H_1}}  &= (H \otimes I) \Ket{00} \\
                   &= (\frac{1}{\sqrt{2}} \begin{pmatrix} 1 & 1 \\ 1 & -1 \end{pmatrix} \otimes \begin{pmatrix} 1 & 0 \\ 0 & 1 \end{pmatrix}) \Ket{00} \\
                    &= \frac{1}{\sqrt{2}} \begin{pmatrix} 1 & 0 & 1 & 0 \\ 0 & 1 & 0 & 1 \\ 1 & 0 & -1 & 0 \\ 0 & 1 & 0 & -1 \end{pmatrix} \Ket{00} \\
                    &= \frac{1}{\sqrt{2}} \begin{pmatrix} 1 & 0 & 1 & 0 \\ 0 & 1 & 0 & 1 \\ 1 & 0 & -1 & 0 \\ 0 & 1 & 0 & -1 \end{pmatrix} \begin{pmatrix} 1 \\ 0 \\ 0 \\ 0 \end{pmatrix} \\
                    &= \frac{1}{\sqrt{2}} \begin{pmatrix} 1 \\ 0 \\ 1 \\ 0 \end{pmatrix} \\
                    &= \Ket{+0}
\end{align}

This corresponds to the intuition that after applying a $H$ gate to the first
qubit, it should be in the $\Ket{+}$ state while the second qubit remains in
the $\Ket{0}$ state.

Similar as for the first gate, the unitary applied to the state when multiple
gates are applied to different qubits at the same time, is constructed by the
tensor product of those gates. The state $\Ket{\psi_{Z_1X_2}}$ after the $Z$
gate is applied to first and the $X$ gate is applied to the second qubit is
calculated as:

\begin{align}
  \Ket{\psi_{Z_1X_2}}  &= (Z \otimes X) \Ket{+0} \\
                    &= \begin{pmatrix} 1 & 0 \\ 0 & -1 \end{pmatrix} \otimes \begin{pmatrix} 0 & 1 \\ 1 & 0 \end{pmatrix}) \Ket{+0} \\
                    &= \begin{pmatrix} 0 & 1 & 0 & 0 \\ 1 & 0 & 0 & 0 \\ 0 & 0 & 0 & -1 \\ 0 & 0 & -1 & 0 \end{pmatrix} \Ket{+0} \\
                    &= \begin{pmatrix} 0 & 1 & 0 & 0 \\ 1 & 0 & 0 & 0 \\ 0 & 0 & 0 & -1 \\ 0 & 0 & -1 & 0 \end{pmatrix} \begin{pmatrix} \frac{1}{\sqrt{2}} \\ 0 \\ \frac{1}{\sqrt{2}} \\ 0 \end{pmatrix} \\
                    &= \begin{pmatrix} 0 \\ \frac{1}{\sqrt{2}} \\ 0 \\ -\frac{1}{\sqrt{2}} \end{pmatrix} \\
                    &= \Ket{-1}
\end{align}

This again matches with the intuition that applying a $Z$ gate to the $\Ket{+}$
state leaves the first qubit in the $\Ket{-}$ state and the $X$ gate on the
second qubit moves it from the $\Ket{0}$ to the $\Ket{1}$ state.

The final state $\Ket{\psi_{final}}$ of the circuit is calculated by applying the $4 \times 4$ matrix
representation of the controlled $H$ gate to $\Ket{-1}$. Notice that the second
qubit is the controlled qubit in this case:

\begin{align}
  \Ket{\psi_{final}}    &= CH \Ket{-1} \\
                        &= \begin{pmatrix} 1 & 0 & 0 & 0 \\ 0 & \frac{1}{\sqrt{2}} & 0 & \frac{1}{\sqrt{2}} \\ 0 & 0 & 1 & 0 \\ 0 & \frac{1}{\sqrt{2}} & 0 & -\frac{1}{\sqrt{2}} \end{pmatrix} \Ket{-1} \\
                       &= \begin{pmatrix} 1 & 0 & 0 & 0 \\ 0 & \frac{1}{\sqrt{2}} & 0 & \frac{1}{\sqrt{2}} \\ 0 & 0 & 1 & 0 \\ 0 & \frac{1}{\sqrt{2}} & 0 &  -\frac{1}{\sqrt{2}} \end{pmatrix} \begin{pmatrix} 0 \\ \frac{1}{\sqrt{2}} \\ 0 \\ -\frac{1}{\sqrt{2}} \end{pmatrix} \\
                       &= \begin{pmatrix} 0 \\ 0 \\ 0 \\ 1 \end{pmatrix} \\
                       &= \Ket{11}
\end{align}

The first qubit has been swaped from the $\Ket{-}$ to the $\Ket{1}$ state as the
second qubis is in the $\Ket{1}$ state and thus the $H$ gate has been applied.
The circuit \ref{fig:circuit1} thus maps the inital state $\Ket{00}$ to $\Ket{11}$.

Another simple yet interesting quantum circuit is shown in figure \ref{fig:circuit2}.

\begin{figure}[h]
  \centering
  \leavemode
  \mbox{
    \Qcircuit @C=1em @R=1em {
      & \lstick{\Ket{0}} & \gate{H} & \ctrl{1} & \qw & \meter{0/1} \\
      & \lstick{\Ket{0}} & \qw & \targ & \qw & \meter{0/1}
    }
  }
  \label{fig:circuit2}
  \caption{Bell creation circuit.}
\end{figure}

The Hadamard gate on the first qubit maps the system from the initial $\Ket{00}$
into the $\Ket{+0}$ state. Afterwards, the first qubit, which is currently
between the $\Ket{0}$ and $\Ket{1}$ state on the Bloch Sphere, is used as the
control qubit in the $CNOT$ gate. This has an interesting effect on the second qubit
as the $X$ gate is applied and not applied at the same time to it, entangling
the two qubits with each other. The final state before measurement is then
$\frac{\Ket{00} + \Ket{11}}{2}$ which has already been discussed in section X.
The state is also known as one of the for \textit{Bell states} which represent maximally
entangled two qubit states and which play an important role in the analysis of
\textit{quantum communication}.

This simple quantum circuit demonstrates how entanglement can be created by the
application of controlled gates with qubits in superposition states.
Entanglement plays an important role in the construction of so called random
quantum circuits in recent quantum supremacy exeperiments. These kind of circuits will
be discussed in chapter X after a short overview of the theoretical
computational power of quantum computers.


\section{Quantum Computational Complexity}

It is important to understand the theoretical capabilities and limitations of quantum computers to understand for which kind of problems
they provide advantages over classical computers. For many years, it has been
assumed that the extended Church Turing thesis, stating that a probablistic
Turing machine can efficiently simulate any realistic model of computation,
holds. This thesis is challenged by quantum computers which potentially can
solve specific problems exponentially faster than classical computers.

Though there is no hard proof yet, there is complexity theoretical evidence that the
extended Church Turing thesis might not hold and that quantum computers can be
realised which can solve specific problems proofable faster than classical
computers. The class of problems which can be solve efficiently by a quantum
computer is called \textbf{BQP}, shorthand for \textit{bounded-error quantum
  polynomial time}. A decision problem is in BQP if there exists a quantum
program that solves the decision problem in 2/3 of the cases and runs in
polynomial time. This is the quantum analoge of the \textit{bounded-error
  probablistic polynomial time} or \textbf{BPP} which is decidable by a
probablistic Turing machine in polynomial time and believed to be the same as
\textbf{P}.

Currently, there are only a few problems known to be in BQP which are suspected
 not to be in P, providing evidence for the superiority of quantum computers. One
of these problems is \textit{factorisation}, the problem to decompose a
composite integer into its prime factors. This problem has been known to be in
\textit{NP} before. It is also known, though, that factorisation is not an NP
hard problem, indicating that quantum computers are probably not able to provide
an exponential speedup for every problem which is not efficiently solvable on a
classical computer. Indeed, classical algorithms might even exist which can
solve factorisation on a classical computer efficiently that have not been
discovered yet.

While BQP seems to include P and intersect with NP it can be shown that it is strictly
included in PSPACE. The relationship of these complexity classes is visualised
in figure \ref{fig:complexityclasses}.

\begin{figure}[h]
  \centering
  \begin{tikzpicture}

    \draw[line width=0.5mm] (orig) circle (2.5);
    \draw[dashed, line width=0.5mm] (orig) ellipse (4.1 and 2.7);
    \draw[line width=0.5mm] (0,1) circle (4);
    \draw[line width=0.5mm] (0,2) circle (5);

    \draw (0,0) node {\large P};
    \draw (0,4) node {\large NP};
    \draw (0,6) node {\large PSPACE};
    \draw (4,-2) node {\large BQP?};

  \end{tikzpicture}
  \caption{Relation of Complexity Classes to each other.}
  \label{fig:complexityclasses}
\end{figure}

While it is hard to proof some fundamental relationships between complexity
classes like the famous $P = NP$ problem, it is believed that quantum computers
can solve specific problems with practical applications like integer factorisation or the simulation of
quantum systems up to exponentially faster than classical computers. The moment
in time a physical quantum computer is able to outperform a classical computer
on a specific problem for the first time has been coined by John Preskill in
2012 as \textit{quantum supremacy}. Recently, Google announced their quantum
supremacy results with a 54 qubit quantum computer, further challenging the
extended Church Turing thesis and providing the first physical evedince that it
might be possible to build quantum computers with advantages over classical
computers in the future.
