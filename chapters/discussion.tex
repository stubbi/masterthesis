\chapter{Conclusion and Discussion}
\label{sec:discussion}

\paragraph{Highlights.}
The results of the experiments show that RBMs are able to accurately simulate random circuit 
instances with 4 qubits and a depth of up to 20 cycles. The comparison of 
Stochastic Reconfiguration (SR) and AdaMax as optimization algorithms suggests that 
AdaMax is better suited for the given gate set and circuit structure with the chosen training 
parameters. On each circuit instance, the best performing training parameters led 
the RBMs reach TVDs below 0.01. Less than 100 training iterations were sufficient to 
apply non-diagonal single-qubit gates to the state represented by the RBMs. 
The more training samples were available, the better the RBMs resembled the target output distributions.

\paragraph{Performance of Stochastic Reconfiguration.}
The high TVDs of RBMs trained with the SR methods might be grounded in the algorithm finding 
local minima. These might be optimal only for a subset of the state space. The oscillation for SR could be explained by the algorithm finding an 
optimum for one training batch, which does not generalize well to the other training samples.
This would also explain why there is no oscillations for SR with restarts on 8 samples. In that
cases, all training samples are inside the same batch. Increasing the batch size or reducing the 
number of hidden units could improve the performance of the SR algorithm.

\paragraph{Effects of Random Restarts.}
Using SR with random restarts had only small if any 
impact on the resulting accuracy of the RBMs. However, the testing and training overlaps 
look different during the training process. Without the restarts, there is not much 
oscillation in these metrics. This supports the assumption that with restarts, the 
RBM is overfitting the current training batch, which does not generalize well to training 
samples in other batches. 

This does not imply that this would also 
hold for AdaMax. Indeed, pre-studies suggested that random restarts might have a positive 
impact on the RBM's accuracy when trained with AdaMax.

\paragraph{Effects of Learning the $CZ$ Gate.}
There was no significant difference in the TVDs between RBMs to which the $CZ$ gates have 
been applied directly and those that learned the $CZ$ gates. At least for the given circuit 
sizes, this implies that adding more hidden units to the RBM during the simulation of 
quantum circuits had no significant effect on the training of the single-qubit non-diagonal 
gates. 

These results suggest that for the given gate set and circuit structure, the $CZ$ gates can 
be applied directly to the RBM's state. This significantly reduces the number of gates which 
have to be applied through a learning phase on the given kind of random circuits.

\paragraph{Performance of AdaMax.}
The training and testing overlaps for AdaMax show that the single-qubit gates can 
be learned within less than 100 training iterations. Applying 
the $CZ$ gates directly will omit the need for more training iterations for the given RBM size.

Further, it seems that the more training samples are available, the better the RBM will 
perform when trained with AdaMax. More experiments will be necessary 
to understand how the number of samples scales with the number of qubits. Establishing
a relation between the number of training samples and the resulting gate fidelity would 
be a major step in understanding the capabilities of RBMs for the classical simulation of 
quantum circuits.

\paragraph{Influence of the Number of Training Samples.}
In the conducted experiments, 303 samples were enough to sample each configuration of the 
$2^n$ dimensional state space multiple times. Only for 8  samples, the full state space
could not be observed during the training. The performance of the RBMs trained with 8 samples
had been very low. Only if RBMs can achieve acceptable performance with a number of samples 
that does not scale exponentially with the number of qubits, this approach would be useful for 
real-world applications. 

In this regard, this study lacks on interpretable insights. Rather than making use of the 
Coupon-collectors problem, it would have been possible to train the RBMs with less samples. 
The training and test sets could have been made up of a defined number of different 
samples. This might have given clearer insights into the relation between the ratio of the 
state space observed and the resulting RBM performance. For future research, the samples should
be drawn that way.

\paragraph{RBMs for the Classical Simulation of NISQ algorithms.}
Noisy simulations with 
fidelities in the range of that from the Sycamore processor could help to understand the 
capabilities and limits of NISQ devices using RBMs on classical computing hardware. 
For 10 qubits, for instance, the cross entropy difference
of the Sycamore processor was estimated to be around 0.4. On the 4 qubit circuits with 
15 cycles, the fidelity of the best performing RBM is at about 0.9. 
If RBMs are able to simulate quantum circuits with a similar accuracy as the Sycamore 
processor on a larger number of qubits with acceptable computational requirements, this 
would make RBMs a suitable simulation framework to test NISQ algorithms.

\paragraph{Cross Entropy Difference as a Performance Metric.}
The cross entropy difference was specifically designed to measure the 
ability of an algorithm or device to sample from the output distributions of 
random circuits. In the conducted experiments, it was also possible to 
calculate the TVD between the RBM's output and the true output distributions of 
the random circuit instances. The results show that a low TVD did not guarantee a 
high cross entropy distance. The reasons for this could be twofold: First, the 
cross entropy distance assumes a depolarizing noise model of the quantum computation. 
Such a model might not be applicable to RBMs. Second, the considered circuits have been 
close to Porter-Thomas, but did not generate perfect Porter-Thomas output distributions.
These two reasons might explain that the cross entropy distance and TVD did not 
always correlate in the conducted experiments. Nevertheless, the cross entropy distance 
was highest for RBMs trained with AdaMax, which also reached the lowest TVDs.

\paragraph{Future Research.}
Understanding the ability of RBMs to simulate random circuits with a wider range of qubits 
was already intended for this work. However, the performed experiments 
on 4 qubit random circuits took already 8 weeks. Testing RBMs with the same 
level of detail on a range of up to 20 qubits was estimated to take up to 
a year. This would have exceeded the time limits of a Master's thesis. Maintenance work on the Noctua Cluster 
after the first batch of experiments made it impossible to conduct more experiments.

Nevertheless, the software developed in the progress of this work is available open source. 
This makes it easy to adapt the code to conduct experiments 
with other ranges of parameters and quantum circuits. For the given kind of random circuit 
sampling experiments, everything is set up to perform more experiments on a broader range of 
qubits. Such experiments could further investigate the empirical relation of TVD and cross entropy 
distance. Additionally, they could help to establish a relation between the number of samples and 
the accuracy of the RBMs' simulations.

The framework can also be easily used to simulate other NISQ algorithms. This could help to 
understand such algorithms when no physical device is available and how the algorithm
behaves with different levels of noise in the computation.

This thesis is the first to apply RBMs to the simulation of random circuit instances. 
The developed software is publicly available on GitHub \cite{NQS2020} and can be used to simulate arbitrary quantum 
circuits provided in the QASM format. The results from 
this thesis give a reference for suitable training parameters which have not been included in related works.


