\chapter{Introduction}

\paragraph{The Relevance of Quantum Computing.}
Computers can be programmed to solve a wide range of problems.
While they can solve some problems efficiently, 
other problems require exponential time or memory. The simulation 
of quantum physical systems is such a problem that requires 
exponential resources \cite{nielsen2002quantum}.  
These simulations could help in the design 
of new materials and in drug discovery \cite{8585034}.

The computational complexity of simulating quantum systems led 
Richard Feynman to propose the concept of a quantum computer \cite{feynman1982simulating} in 1982. 
Quantum computers work fundamentally different from classical computers by 
harnessing quantum physics for computation. Large-scale quantum computers have the potential
to simulate the physics of quantum systems efficiently \cite{Zalka_1998}.

Since Feynman's proposal in 1982, researchers are making progress in building small-scale 
quantum devices and understanding their computational capabilities and limitations.

Among the discoveries of the potential of the computation power of quantum computers is Shor's algorithm \cite{shor1997factorisation}: 
In 1994, Peter Shor discovered a quantum algorithm that is able to solve the 
factorization problem in polynomial time. This makes quantum computers a potential thread to 
today's internet security protocols. Most of these protocols are based on the assumption 
that the factorization problem cannot be solved efficiently, and that it would take thousands of years to find the 
prime factors of the public encryption keys.

\paragraph{Near-Term Quantum Computing and Quantum Supremacy.}
Even though error-corrected large-scale quantum computers are probably still 
years away, so-called noisy-intermediate scale 
quantum (NISQ) devices represent the first generation of quantum devices.
Studying their capabilities and limits could enable us to harness possible advantages 
of quantum computers already today. Moreover, it might be of further help understanding
the limits of large scale quantum computers \cite{Preskill_2018}.

In 2019, researchers from Google and the University of Santa Barbara (UCSB), California, built a 
53 qubit quantum processor, called Sycamore. They performed a calculation on it that they claimed could not be 
performed on a classical computer \cite{martines2019supremacy}. This moment was highly anticipated 
as "quantum supremacy", a term coined by John Preskill in 2012 \cite{preskill2012quantum}. In the experiments, 
samples from random quantum circuits were drawn with the Sycamore processor.
The random circuit sampling experiment was specifically tailored 
to the demonstration of quantum supremacy \cite{Boixo2018supremacy}.

\paragraph{Machine Learning for the Classical Simulation of Quantum Computing.}
To understand the capabilities of such NISQ devices, one also has to understand 
the limits of classical computation. Following the recent success of machine learning in 
many scientific and industrial domains, in 2017, Carleo and Troyer applied restricted Boltzmann machines (RBMs),
one class of recurrent neural networks,
to simulate wave functions of different many-body quantum systems classically \cite{carleo2017solving}.
Their results showed that RBMs are able to offer a compact representation of the wave functions 
of different many-body quantum systems. In 2018, 
J\'{o}nsson et al adapted the framework to simulate small quantum programs with 
RBMs \cite{jnsson2018neuralnetwork}. Recently, Medvidovic and Carleo used the 
same approach for a noisy simulation of the quantum approximate optimization algorithm 
(QAOA) with up to 53 qubits \cite{medvidovic2020classical}.

\paragraph{Scope of the Thesis.}
This thesis investigates the abilities of RBMs to simulate the random circuit sampling 
experiments conducted in Google's supremacy experiments on the Sycamore processor \cite{martines2019supremacy}. 
The objective of this thesis is to examine if RBMs are able to sample from the output distribution 
of small random circuit instances.

Stochastic Reconfiguration (SR) \cite{sorella1998green} and 
AdaMax \cite{kingma2014adam} are compared as two different optimization algorithms for the 
adaption of the RBM's parameters. Both methods have successfully been used before to train RBMs 
for the representation of quantum system states \cite{jnsson2018neuralnetwork}, \cite{carleo2017solving}, \cite{carleo2018constructing}, \cite{medvidovic2020classical}.
A direct comparison of both methods is yet missing. 

Additionally, a range of training samples and training iterations 
is tested to gain insights into the training process and the required computational resources. 
These parameters have not been reported in related works. One goal of this thesis is 
to act as a reference for suitable training parameters. 

Total variation distance (TVD) and the cross entropy difference are used as 
performance metrics. This allows for an empirical comparison of TVD and cross entropy difference 
on small random circuits.

Understanding the
fidelities with which RBMs can simulate quantum circuits could help to understand
the limits of classical computation. The software developed as part of this thesis 
might further help to study algorithms designed for NISQ devices. To support 
further research, it is published as 
an open source library on GitHub \cite{NQS2020}.
The experiments are conducted on the Noctua cluster located at the 
University of Paderborn, Germany \cite{noctua2020}.

\paragraph{Structure of the Thesis.}
This thesis consists of three parts: 
In the first part (chapter~\ref{sec:notation} to ~\ref{sec:rbm}), the mathematical 
notation is introduced and the theoretical framework is explained. In the 
following part (chapter~\ref{sec:experiments}), 
the experimental setup is described and the results are presented. 
In the last part (chapter~\ref{sec:discussion}), the results are summarized and discussed.
Furthermore, ideas for future research are given.

Chapter~\ref{sec:notation} introduces the mathematical notation used throughout this 
thesis.

In chapter~\ref{sec:quantum_computing}, the foundations of quantum computing are introduced. 
Starting with a single qubit in section~\ref{sec:qubits}, the mathematical framework of multi-qubit systems is derived
in section~\ref{sec:multiplequbitsandentanglement}. The notation and applications 
of quantum gates are explained in section~\ref{sec:quantum_gates}, before small quantum circuits are shown
in section~\ref{sec:quantum_circuits}. The chapter closes with an overview of quantum complexity 
theory in section~\ref{sec:quantum_computational_complexity}.

Chapter~\ref{sec:rcs} explains the concepts of random circuit sampling as a quantum supremacy experiment.
First, the definition of quantum supremacy and different proposals for possible experiments 
are given in section~\ref{sec:quantum_supremacy}. 
In section~\ref{sec:cross_entropy} and~\ref{sec:supremacy_regime}, the 
cross entropy difference, the main metric of random circuit sampling, is derived.
The structure of the considered random circuits is explained in section~\ref{sec:random_circuit_design}.
In the last part of the chapter (section~\ref{sec:experiments_sycamore}), the supremacy experiment conducted by Google and the UCSB 
and its results are discussed.

In chapter~\ref{sec:rbm}, the concept of the Boltzmann machine is introduced. 
Starting with general Boltzmann machines in section~\ref{sec:gbm}, restricted 
Boltzmann machines are explained in section~\ref{sec:rbms}.
Sections~\ref{sec:gibbsSampling} and ~\ref{sec:learning} explain how 
Boltzmann machines can be trained to resemble the probability distribution implied by a given set of samples.
The chapter closes with section~\ref{sec:applicationToQuantumComputing}, which explains
how restricted Boltzmann machines can be applied to the simulation of quantum circuits.

The experiments and obtained results are described in chapter~\ref{sec:experiments}. 
The experimental setup is detailed in section~\ref{sec:setup}. The results 
for the different training strategies are presented in section~\ref{sec:results}.

The thesis closes with a discussion of the results in chapter~\ref{sec:discussion}.
Highlights and limitations of the experiments are described. 
Directions for future research are proposed.


