\chapter{Introduction}

Computers can be programmed to solve a wide range of problems.
While they can solve some problems efficiently, 
other problems require exponential time or memory. One example of 
a problem that cannot be solved efficiently that has real-world 
applications is the simulation of quantum systems \cite{nielsen2002quantum}.  
Such simulations could help in the design 
of new materials and in drug discovery \cite{8585034}.

The computational complexity of simulating quantum systems led 
Richard Feynman in 1982 to propose the concept of a quantum computer \cite{feynman1982simulating}. Quantum 
computers work fundamentally different from classical computers by 
harnessing quantum physics for computation. Large-scale quantum computers have the potential
to simulate the physics of quantum systems efficiently \cite{Zalka_1998}.

Since Feynman's proposal in 1982, researchers are making progress in building small-scale 
quantum devices and in understanding their computational capabilities and limitations.
Among the discoveries of the potential of quantum computing is Shor's algorithm \cite{shor1997factorisation}: 
In 1994, Peter Shor gave a quantum algorithm that is able to solve the 
factorization problem efficiently. This makes quantum computers a potential thread to 
today's internet security protocols. Most of these protocols are based on the assumption 
that the factorization problem cannot be solved efficiently and that it would take thousands of years to find the 
prime factors of the public encryption keys.

\paragraph{Near-term Quantum Computing and Quantum Supremacy.}
Even though error-corrected large-scale quantum computers are probably still 
years away, so-called noisy-intermediate scale 
quantum (NISQ) devices represent the first generation of quantum devices.
Studying their capabilities and limits could enable us to harness possible advantages 
of quantum computers already today and further understand the limits of 
large scale quantum computers \cite{Preskill_2018}.

In 2019, researchers from Google and the UCSB built a 
53 qubit quantum processor and performed a calculation on it that they claimed could not be 
performed on a classical computer \cite{martines2019supremacy}. This moment was already highly anticipated 
as "quantum supremacy", a term coined by John Preskill in 2012 \cite{preskill2012quantum}. In their experiments, 
the quantum circuit calculates the output probabilities of samples taken from 
random quantum circuits. The underlying problem statement is specifically tailored 
to the demonstration of quantum supremacy \cite{Boixo2018supremacy}.

\paragraph{Machine Learning for the Classical Simulation of Quantum Computing.}
To understand the capabilities of such NISQ devices, one also has to understand 
the limits of classical computation. Following the recent success of machine learning in 
many domains, in 2017 Carleo and Troyer applied restricted Boltzmann machines (RBMs)
to simulate the wave-function of many-body quantum systems classically \cite{carleo2017solving}.
Their results showed that RBMs are able to give a compact representations of the wave functions 
of different many-body quantum systems. In 2018, 
J\'{o}nsson et al adapted the framework to simulate small quantum programs with 
RBMs \cite{jnsson2018neuralnetwork}. Recently, Medvidovic and Carleo used the 
same approach for a noisy simulation of the QAOA algorithm with up to 53 qubits \cite{medvidovic2020classical}.

\paragraph{Structure of this Thesis.}
This study investigates the abilities of RBMs to simulate the random circuit sampling 
experiments conducted in Google's experiments on the Sycamore processor. Understanding the
fidelities with which RBMs can simulate random circuits might help to understand
the limits of classical computation. The software developed as part of this thesis 
might further help to study algorithms designed for NISQ devices. It is published as 
an open source library at \cite{NQS2020}.

This study is structured as follows: First, a convention on the used mathematical 
symbols is given.

In chapter~\ref{sec:quantum_computing}, quantum computing is introduced. Starting with a single qubit,
the mathematical framework of many-quantum systems is derived. The notion and application 
of quantum gates is explained before sample quantum circuits are shown.

Chapter~\ref{sec:rcs} about random circuit sampling builds up on the introduction into quantum computing.
The definition of quantum supremacy and different proposals to demonstrate it are given. The 
cross entropy fidelity, the main metric in the random circuit sampling experiments, and how 
it translates into the quantum supremacy regime is mathematically justified. The design of 
the random circuits is explained and an algorithm to generate them is described. In the last
part of the chapter, the experiments and their results by the teams of Google and the UCSB on 
their Sycamore quantum processor are discussed.

In chapter~\ref{sec:rbm}, the concept of the Boltzmann machine is introduced. General as well as restricted 
Boltzmann machines, as they are used in the experiments in chapter~\ref{sec:experiments}, are explained. The chapter further 
outlines how the parameters of a restricted Boltzmann machine can be adapted in an iterative 
training process such that samples drawn from the RBM are distributed according to probability 
distributions implied by a set of training samples. The last part of the chapter details how 
RBMs can be used for the classical simulation of quantum circuits.

After the theoretical framework has been laid out, the experiments are described in chapter~\ref{sec:experiments}. Details of 
the experimental setup are given and the results of the different training strategies for the 
RBM are shown.

In the last two chapters, the results are discussed. Highlights and limitations of the 
experiments and their results are described. Directions for future research are proposed.
