\chapter{Introduction}

Programmable computers are able to solve a wide range of problems.
While they can solve some of these problems efficiently, 
other problems require exponential time or memory. One example of 
a problem that cannot be solved efficiently with real-world 
applications is the simulation of quantum systems. Simulating 
the physics of many-body quantum systems would help in the design 
of new materials and drug discovery.

The computational complexity of simulating quantum systems 1981 led 
Richard Feyman to propose the concept of a quantum computer. Quantum 
computers work fundamentally different from classical computers by 
harnessing quantum physics for computation. Quantum computers would be 
able to simulate quantum systems efficiently.

Since then, researchers are trying to physically build quantum devices
and to understand their computational capabilities and limitations. In 1994, 
Peter Shor could prove that a quantum computer is able to solve the 
factorization problem efficiently. While most of today's internet security 
is based on the assumption, that it takes thousands of years to find the 
prime factors of big integers, quantum computers could break these 
security protocols.

Last year, researchers from Google and the UCSB claimed to have built a 
53 qubits quantum processor and performed a calculation on it that could not be 
performed on a classical computer. This moment was already highly anticipated 
and coined "quantum supremacy" by John Preskill in 2012. In their experiments, 
the quantum circuit calculates the output probabilities of samples taken from 
random quantum circuits. The underlying problem statement is specifically tailored 
to the demonstration of quantum supremacy.

Even though error-corrected large-scale quantum computers are probably still 
years away, so-called noisy-intermediate scale 
quantum (NISQ) devices represent the first generation of quantum devices.
Studying their capabilities and limits could enable us to harness possible advantages 
of quantum computers already today and further understand the limits of 
large scale quantum computers.

To understand the capabilities of such NISQ devices, we also have to understand 
the limits of classical computation. The recent success of machine learning in 
many domains lead Carleo and Troyer in 2016 to apply restricted Boltzmann machines (RBMs)
to simulate the wave-function of many-body quantum systems classically. In 2018, 
Johnson et al adapted the framework to simulate small quantum programs with 
RBMs. Recently, Carleo and X used the same approach to simulate the QAOA algorithm 
with up to 53 qubits.

This study investigates the abilities of RBMs to simulate the random circuit sampling 
experiments conducted last year on the Sycamore processor. Understanding the
fidelities with which RBMs can simulate random circuits might help to understand
the limits of classical computation. The software developed as part of this thesis 
might further help to study algorithms designed for NISQ devices. It is published as 
an open source library.

This study is structured as follows: First, a convention on the used mathematical 
symbols is given.

In the following chapter, quantum computing is introduced. Starting with a single qubit,
the mathematical framework of many-quantum systems is derived. The notion and application 
of quantum gates is explained before sample quantum circuits are shown.

The chapter about random circuit sampling builds up on the introduction into quantum computing.
The definition of quantum supremacy and different proposals to demonstrate it are given. The 
Cross-Entropy fidelity, the main metric in the random circuit sampling experiments, and how 
it translates into the quantum supremacy regime is mathematically justified. The design of 
the random circuits is explained and an algorithm to generate them is described. In the last
part of the chapter, the experiments and their results by the teams of Google and the UCSB on 
their Sycamore quantum processor are discussed.

In chapter 5, the concept of the Boltzmann machine is introduced. General as well as restricted 
Boltzmann machines, as they are used in the experiments, are explained. The chapter further 
lines out how the parameters of a restricted Boltzmann machine can be adapted in an iterative 
training process such that samples drawn from the RBM are distributed according to probability 
distributions implied by a set of training samples. The last part of the chapter details how 
RBMs can be used for the classical simulation of quantum circuits.

After the theoretical framework has been laid out, the experiments are described. Details of 
the experimental setup are given and the results of the different training strategies for the 
RBM are shown.

In the following chapter, the results are discussed. Highlights and limitations of the 
experiments and their results are described. Directions for future research are proposed.
