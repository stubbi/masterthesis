\chapter{Summary}

This work gives an introduction into the field of quantum computing, lays out the theoretical framework 
of random circuit sampling, and explains the math of restricted Boltzmann machines. RBMs are further used 
in this study to simulate random circuit instances. Different training parameters are varied to understand their 
influence on the RBMs capabilites to simulate random quantum circuits of different sizes. 

The results of this work confirm that RBMs can be used as an alternative approach to the classical 
simulation of quantum circuits. When trained with AdaMax and random restarts, the best performing RBMs
have been able to approximate the resulting quantum state of 4-qubit random circuit instances with depth 
of up to 20 cycles accurately. 

Nevertheless, the performance of RBMs should be further verified on random circuit instances with more qubits. 
For larger system sizes, the influence of the number of training iterations and training samples has to be further 
analyzed. This will help to understand the computational resources needed to simulate quantum circuits 
with a certain fidelity. Such insights would help to better understand the current state of the physical 
implementations of quantum computers as RBMs could be used as a benchmark for NISQ devices (REALLY?).

The software developed for this study is available as open source at \cite{}. It can easily be adapted to 
train RBMs on different random circuit instances with varying system sizes on the Noctua Cluster or any other 
computing hardware. It further allows the easy adaption of circuits on which the RBMs should be trained to 
analyze not only random circuits but also other classes of quantum circuits in the future. The results of this 
study might guide further research and suggest good choices of training parameters which can be further studied and 
optimized.
