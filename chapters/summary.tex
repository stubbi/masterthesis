\chapter{Summary}

This thesis gives an introduction into the field of quantum computing, lays out the theoretical framework 
of random circuit sampling, and explains restricted Boltzmann machines (RBMs). RBMs are further used 
to simulate random circuit instances classically. Different training parameters are varied to understand their 
influence on the RBMs capabilities to simulate random quantum circuits of different sizes. 

The outcomes confirm prior results that RBMs are able to give compact representations of quantum circuits. 
They further proof that RBMs are able to sample from the output distributions of random circuits. 
When trained with AdaMax and random restarts, the best performing RBMs
have been able to approximate the resulting quantum state of 4-qubit random circuit instances with a depth 
of up to 20 cycles accurately. 

Nevertheless, the performance of RBMs should be further studied on random circuit instances with more qubits. 
For larger system sizes, the influence of the number of training iterations and training samples has to be further 
analyzed. This will help in the understanding of the computational resources required to simulate quantum circuits 
with targeted fidelities. Such insights would help to better understand the current state of physical 
implementations of quantum computers. Further, RBMs could be used to simulate and analyse NISQ algorithms
when no physical device is available.

The software developed for this study is available as open source at \cite{NQS2020}. It can easily be adapted to 
train RBMs on random circuit instances with varying system sizes on the Noctua Cluster or any other 
computing hardware. It further provides an interface to simulate any quantum circuit given in the QASM format.

The results of this thesis suggest good choices of training parameters which should be further studied and 
optimized.
