\chapter{Nomenclature and Notation}
\label{sec:notation}

This chapter introduces mathematical notations and expressions used throughout 
this study. Its focus is on linear algebra, the mathematical foundation of 
\textit{quantum mechanics}. It is thought as a lookup reference for the following chapters.

Vectors are represented in the \textit{Bra-ket} notation. \textit{Ket}-vectors $\Ket{\psi}$ denote 
column vectors:

\begin{equation}
   \Ket{\psi} = \begin{pmatrix} \alpha \\ \beta \end{pmatrix}.
\end{equation}

The \textit{bra} $\Bra{\psi}$ of a vector $\Ket{\psi}$ is its hermitian conjugate, 
$\Bra{\psi} = \Ket{\psi}^{\dagger}$. It is the transpose of the vector with complex-conjugated 
entries. For the ket-vector $\Ket{\psi}$ above, the corresponding bra-vector would be:

\begin{equation}
   \Bra{\psi} = \begin{pmatrix}
      \alpha^* & \beta^*
   \end{pmatrix},
\end{equation}

where the complex-conjugate of some complex number $\alpha = a + b i$ is 
defined as $\alpha^* = a - b i$. Using this notation, the \textit{inner product} of two vectors
$\Ket{\psi}$ and $\Ket{\phi}$ can be written as:

\begin{equation}
   \Braket{\psi | \phi} = \sum_j \psi_j^* \phi_j.
\end{equation}

The \textit{outer product} $\Ket{\psi} \Bra{\phi}$ defines the matrix $A$ with entries 
$a_{ij}$ given by:

\begin{equation}
   a_{ij} = \psi_i \phi^*_j.
\end{equation}

The class of matrices of special interest in quantum computing are \textit{unitary} matrices.
A complex-valued square matrix $U$ is unitary if:

\begin{equation}
   U U^{\dagger} = U^{\dagger}U = I.
\end{equation}

Unitary matrices are norm preserving and thus represent rotations in the vector space.
The \textit{tensor product} $\cdot \otimes \cdot$ of two matrices $A=\begin{pmatrix}
   a_{11} & a_{12} \\ a_{21} & a_{22}
\end{pmatrix}$ and $B= \begin{pmatrix}
   b_{11} & b_{12} \\ b_{21} & b_22
\end{pmatrix}$ is defined as:

\begin{align}
   A \otimes B &= \begin{pmatrix}
      a_{11} B & a_{12} B \\ a_{21} B & a_{22} B
   \end{pmatrix} \\
   &= \begin{pmatrix}
      a_{11} b_{11} & a_{11} b_{12} & a_{12} b_{11} & a_{12} b_{12} \\
      a_{11} b_{21} & a_{11} b_{22} & a_{12} b_{21} & a_{12} b_{22} \\ 
      a_{21} b_{11} & a_{21} b_{12} & a_{22} b_{11} & a_{22} b_{12} \\
      a_{21} b_{21} & a_{21} b_{22} & a_{22} b_{21} & a_{22} b_{22}
   \end{pmatrix}.
\end{align}

The dimensions of the tensor product of two matrices with dimensions $n_A \times m_A$ and $n_B \times m_B$ 
is $n_An_B \times m_Am_B$. The tensor product for vectors is defined accordingly.
The \textit{Kroenecker Delta} $\delta_{ij}$ sometimes occurs in quantum physics. It is defined as:

\begin{equation}
   \delta_{ij} =
    \begin{cases}
            1, &         \text{if } i=j,\\
            0, &         \text{if } i\neq j,
    \end{cases}.
\end{equation}

and by that defines the dot product of two vectors $\Ket{\psi_i}$ and $\Ket{\psi_j}$ from 
a set of orthonormal vectors.



