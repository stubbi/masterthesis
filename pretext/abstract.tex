\chapter*{\centering \begin{normalsize}Abstract\end{normalsize}}

Random circuit sampling provides a benchmarking tool for noisy intermediate-scale quantum (NISQ)
devices and simulations.
This thesis is the first to study the abilities of restricted Boltzmann machines (RBMs)
to sample from the output distribution of such random quantum circuits. 
Variations of Stochastic Reconfiguration and AdaMax are compared for a range of training samples 
and iterations. Total variation distance (TVD) and cross entropy difference are measured 
and compared as performance metrics.
The results show that RBMs are able to simulate random quantum circuits with 4 qubits 
and a depth of up to 20 cycles accurately. When trained with AdaMax, the best 
performing RBMs were able to approximate the average true output distribution with a TVD of 0.00 
on all circuit depths.
The non-diagonal single-qubit gates can be applied within less than 100 training iterations. 
The accuracy of the RBMs had been higher the more training samples were available.
The software developed in the progress of this thesis is publicly available on GitHub. 
It can be adapted to simulate random circuits with a wider range of qubits. It 
further provides a simple interface to run quantum circuits given in the QASM format. This thesis
can act as a reference for suitable training parameters of the RBMs, which have not been 
included in related works.